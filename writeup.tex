\documentclass{article}
\usepackage{graphicx} % Required for inserting images\
\usepackage{amsmath}
\usepackage{amsthm}
\usepackage{algorithm}
\usepackage{algpseudocode}
\usepackage{amsfonts}

\newtheorem{definition}{Definition}

\title{Trajectory Optimization in Smoothed Representation Spaces}
\author{Misha Lvovsky}
\date{November 2023}

\begin{document}

\maketitle

\section{Problem Statement}

We want to create a smooth representation space which is analogous to our state and action spaces.
Let's define that a neighborhood in our representation space has a size of \(1\) unit.
We also define that all states in a neighborhood should be transformed similarly in state space by actions in a neighborhood.

\section{Spaces}

\begin{definition} [State Space]
Let $\mathcal{S}$ signify the space of all possible states for the system.
\end{definition}
\begin{definition} [Action Space]
Let $\mathcal{A}$ signify the action space
\end{definition}
\begin{definition} [Embedding Space]
Let $\mathcal{Z}_s$ and $\mathcal{Z}_a$ be the spaces of all possible embeddings of states and actions respectively.
\end{definition}

\section{Transformations}

\subsection*{Definitions}

\begin{definition} [Encoders]
    Let \(\mathcal{E}\) represent the encoders which map \(\mathcal{E}_s: \mathcal{S}\rightarrow\mathcal{Z}^s\) and \(\mathcal{E}_a: \mathcal{A} \times \mathcal{Z}^s \rightarrow \mathcal{Z}^a\)
\end{definition}
\begin{definition} [Decoders]
    Let \(\mathcal{D}\) represent the decoders model which map \(\mathcal{D}_s: \mathcal{Z}_s\rightarrow\mathcal{S}\) and \(\mathcal{Z}_a \times \mathcal{Z}_s\rightarrow \mathcal{A}\)
\end{definition}
\begin{definition} [Forward model]
    Let \(\mathcal{F}\) represent the forward model which maps \(\mathcal{F}: \mathcal{Z}_s \times \mathcal{Z}_a \rightarrow \mathcal{Z}_s\) where the inputs are state and action at time step \(t\) and the output is the state at time \(t+1\)
\end{definition}
\begin{definition} [Distance Metric]
    Let \(d\) represent a distance metric in the latent state and action spaces \(d_s: \mathcal{Z}_s \times \mathcal{Z}_s \rightarrow \mathbb{R}\) and \(d_a: \mathcal{Z}_a \times \mathcal{Z}_a \rightarrow \mathbb{R}\)
\end{definition}

\subsection*{Details}
\begin{itemize}
    \item The action encoder is conditioned on state because actions depend on context.
    \item When writing the encoder, decoder, and distance metric \(\mathcal{E}\), \(\mathcal{D}\), and \(d\) the subscript will be omitted to simplify notation.
\end{itemize}

\section{Representation Properties}

To create an abstract representation of states and actions which are smooth, we search for a fixed number of state and action neighborhoods which obey the several constraints.

\subsection*{Smoothness Constraint}

The purpose of this loss is to constrain the state and action spaces to be smooth.
States and actions are defined to be in a neighborhood when if the distance metric evaluated on them is less than \(1\) i.e. \(z_i\) and \(z_j\) are in the same neighborhood if \(d(z_i, z_j) < 1\).\\

\noindent
The smoothness loss constrains that actions in the same neighborhood must affect states in a neighborhood similarly (the successor states must also be within the same neighborhood).\\

\noindent
Formally for any state action pairs \((s_i,a_i)\) and \((s_j,a_j)\) the next states \(T(s_i, a_i)\) and \(T(s_j, a_j)\) should be in the same neighborhood if \(s_i\) and \(s_j\), and \(a_i\) and \(a_j\) are in the same neighborhoods.
\begin{align*}
     & \forall (s_i, a_i) \in \mathcal{S} \times \mathcal{A},                            \\
     & \forall (s_j, a_j) \in \mathcal{S} \times \mathcal{A}:                            \\
     & \qquad d(s_i, s_j) < 1 \land d(a_i, a_j) < 1 \iff d(T(s_i, a_i), T(s_j, a_j)) < 1
\end{align*}

\subsection*{Radius Constraint}

The purpose of this constraint is to limit the number of state and action neighborhoods.
This constraint is enforced by setting a radius for the latent state and action spaces \(r_s, r_a\) and defining that no two states or actions can be farther than this radius apart in the latent space.

Formally for any state or action pairs \(s_i, s_j\) or \(a_i, a_j\) we enforce that \(d(s_i, s_j) < r_s\) and \(d(a_i, a_j) < 1\)
\begin{align*}
     & \forall (s_i, s_j) \in \mathcal{S} \times \mathcal{S}, \\
     & \forall (a_i, a_j) \in \mathcal{A} \times \mathcal{A}: \\
     & \qquad d(s_i, s_j) < r_s \land d(a_i, a_j) < r_a
\end{align*}


\subsection*{Dispersion Objective}

The purpose of this objective is to encourage the action space to cover the next state possibilities evenly.\\

\noindent
This is done by encouraging actions to be close together if they affect states similarly, allowing other actions to fill the gap.
This objective complements the smoothness constraint which enforces that actions in a neighborhood will affect states similarly but does not enforce that actions in different neighborhoods will affect states differently.\\

\noindent
Formally the expected value of the difference between the distance between two latent actions and their respective successor latent states from the same source state over a uniform distribution in the latent action space should be minimized.
\[\min\limits\underset{z_{a_i}, z_{a_j} \in \text{Uniform}(\mathcal{Z}_a),\; s \sim \mathcal{D}}{\mathbb{E}}\left[|d\big(F(\mathcal{E}(s), z_{a_i}), \mathcal{F}(\mathcal{E}(s), z_{a_j})\big) - d(z_{a_i}, z_{a_j})|\right]\]

\section{Loss Functions}

\section{Effects}

If we have a trajectory guess as latent actions \(z_{a_{1:T}}\) and a cost function \(j\) which is Lipschitz continuous in the state space.
Under the assumption that the jacobian of the encoder has a determinant less than \(1\) then then if the Lipschitz constant of \(j\) in the state space is \(L\) then the Lipschitz constant in latent state space is \(L_z < L\).\\

\noindent
Additionally since the encoder has the smoothness constraint enforced, we know that if we update only one latent action in our action sequence within the same neighborhood, the successor states will remain in the same neighborhood.
Thus if there is a

\end{document}
